% Options for packages loaded elsewhere
\PassOptionsToPackage{unicode}{hyperref}
\PassOptionsToPackage{hyphens}{url}
%
\documentclass[
]{article}
\usepackage{amsmath,amssymb}
\usepackage{lmodern}
\usepackage{iftex}
\ifPDFTeX
  \usepackage[T1]{fontenc}
  \usepackage[utf8]{inputenc}
  \usepackage{textcomp} % provide euro and other symbols
\else % if luatex or xetex
  \usepackage{unicode-math}
  \defaultfontfeatures{Scale=MatchLowercase}
  \defaultfontfeatures[\rmfamily]{Ligatures=TeX,Scale=1}
\fi
% Use upquote if available, for straight quotes in verbatim environments
\IfFileExists{upquote.sty}{\usepackage{upquote}}{}
\IfFileExists{microtype.sty}{% use microtype if available
  \usepackage[]{microtype}
  \UseMicrotypeSet[protrusion]{basicmath} % disable protrusion for tt fonts
}{}
\makeatletter
\@ifundefined{KOMAClassName}{% if non-KOMA class
  \IfFileExists{parskip.sty}{%
    \usepackage{parskip}
  }{% else
    \setlength{\parindent}{0pt}
    \setlength{\parskip}{6pt plus 2pt minus 1pt}}
}{% if KOMA class
  \KOMAoptions{parskip=half}}
\makeatother
\usepackage{xcolor}
\usepackage[margin=1in]{geometry}
\usepackage{graphicx}
\makeatletter
\def\maxwidth{\ifdim\Gin@nat@width>\linewidth\linewidth\else\Gin@nat@width\fi}
\def\maxheight{\ifdim\Gin@nat@height>\textheight\textheight\else\Gin@nat@height\fi}
\makeatother
% Scale images if necessary, so that they will not overflow the page
% margins by default, and it is still possible to overwrite the defaults
% using explicit options in \includegraphics[width, height, ...]{}
\setkeys{Gin}{width=\maxwidth,height=\maxheight,keepaspectratio}
% Set default figure placement to htbp
\makeatletter
\def\fps@figure{htbp}
\makeatother
\setlength{\emergencystretch}{3em} % prevent overfull lines
\providecommand{\tightlist}{%
  \setlength{\itemsep}{0pt}\setlength{\parskip}{0pt}}
\setcounter{secnumdepth}{-\maxdimen} % remove section numbering
\ifLuaTeX
  \usepackage{selnolig}  % disable illegal ligatures
\fi
\IfFileExists{bookmark.sty}{\usepackage{bookmark}}{\usepackage{hyperref}}
\IfFileExists{xurl.sty}{\usepackage{xurl}}{} % add URL line breaks if available
\urlstyle{same} % disable monospaced font for URLs
\hypersetup{
  pdftitle={Resumen\_10032022.git},
  pdfauthor={Niklas Hess},
  hidelinks,
  pdfcreator={LaTeX via pandoc}}

\title{Resumen\_10032022.git}
\author{Niklas Hess}
\date{2022-10-04}

\begin{document}
\maketitle

``Las entidades {[}\ldots{]} \texttt{aseguradoras} deberán contar con un
departamento o servicio de atención al cliente. Además podrán designar
un defensor del cliente, a quien corresponderá atender y resolver los
tipos de reclamaciones que determine en cada caso su reglamento de
funcionamiento, y que habrá de ser una entidad o experto
independiente.''

\hypertarget{obligaciones-de-las-entidades-aseguradoras-resumen}{%
\section{\texorpdfstring{1. Obligaciones de las entidades aseguradoras
(\emph{Resumen})}{1. Obligaciones de las entidades aseguradoras (Resumen)}}\label{obligaciones-de-las-entidades-aseguradoras-resumen}}

\begin{itemize}
\tightlist
\item
  Atender y resolver las quejas y reclamaciones de sus clientes
  (\textbf{Artículo 3}) en un plazo de 2 meses desde la presentación de
  la queja (\textbf{Artículo 10.3}).
\item
  Disponer de un
  \texttt{departamento\ o\ servicio\ específicamente\ para\ atención\ al\ cliente}
  (\emph{``Departamento''}) (\textbf{Artículo 4}).

  \begin{itemize}
  \tightlist
  \item
    No es obligación, pero sí un derecho disponer de un
    \texttt{defensor\ del\ cliente} (\emph{``Defensor''})
    (\textbf{Artículo 4.2}), al que se le aplicarán los mismos deberes
    que a los departamentos.
  \end{itemize}
\item
  Tener titulares del \texttt{Departamento} (\emph{``Titulares''}) que
  cumpla con los requisitos personales (\textbf{Artículo 5}).

  \begin{itemize}
  \tightlist
  \item
    Que el consejo de administración o equivalente de la entidad
    aseguradora designe a los \texttt{Titulares}.
  \item
    Que la designación se comunique al
    \texttt{Comisionado\ para\ la\ Defensa\ del\ Cliente\ de\ Servicios\ Financieros}
    (\emph{``Comisionado''}) y autoridades supervisoras.
  \end{itemize}
\item
  Implementar las medidas oportunas para separar el
  \texttt{Departamento} de los demás servicios comerciales
  (\textbf{Artículo 6}).

  \begin{itemize}
  \tightlist
  \item
    Que las medidas respeten los principios de rapidez, seguridad,
    eficacia y coordinación.
  \end{itemize}
\item
  Aprobar un Reglamento para la Defensa del Cliente, con su contenido
  oportuno (\textbf{Artículo 8}).
\item
  Poner a disposición del cliente toda información que se establece en
  la Orden (\textbf{Artículo 9}).
\item
  En el caso de queja (\textbf{Artículo 12}):

  \begin{itemize}
  \tightlist
  \item
    Se deberá informar al reclamante sobre la instancia competente para
    conocer su queja.
  \item
    Se requerirá completar la documentación en el plazo de 10 días
    cuando haya un error en la queja.
  \item
    Finalizar el expediente, motivado, en el plazo de 2 meses desde que
    se presentase la queja (\textbf{Artículo 15}).
  \item
    Notificar la decisión en un plazo de 10 días naturales desde la
    fecha de la decisión.
  \end{itemize}
\item
  Atender a los requerimientos que efectúen los \texttt{Comisionados}
  (\textbf{Artículo 16}).
\item
  Establecer acuerdos con los \texttt{Comisionados} para facilitar la
  transmisión de datos y documentos.
\item
  Los \texttt{Departamentos} deberán redactar un informe anual en el
  primer trimestre de cada año explicando el desarrollo de su función,
  cumpliendo con los requisitos de contenido mínimo (\textbf{Artículo
  17}).
\end{itemize}

\hypertarget{requisitos-y-deberes-capuxedtulo-ii}{%
\section{\texorpdfstring{2. Requisitos y deberes (\emph{Capítulo
II})}{2. Requisitos y deberes (Capítulo II)}}\label{requisitos-y-deberes-capuxedtulo-ii}}

\hypertarget{deberesobligaciones-de-las-aseguradoras}{%
\subsection{2.1 Deberes/obligaciones de las
aseguradoras}\label{deberesobligaciones-de-las-aseguradoras}}

\hypertarget{atender-y-resolver-las-quejas-y-reclamaciones-de-sus-clientes-artuxedculo-3}{%
\paragraph{\texorpdfstring{Atender y resolver las quejas y reclamaciones
de sus clientes (\textbf{Artículo
3}):}{Atender y resolver las quejas y reclamaciones de sus clientes (Artículo 3):}}\label{atender-y-resolver-las-quejas-y-reclamaciones-de-sus-clientes-artuxedculo-3}}

\hypertarget{disponer-de-un-departamento-o-servicio-especuxedficamente-para-atenciuxf3n-al-cliente-artuxedculo-4}{%
\paragraph{\texorpdfstring{Disponer de un departamento o servicio
específicamente para atención al cliente (\textbf{Artículo
4}):}{Disponer de un departamento o servicio específicamente para atención al cliente (Artículo 4):}}\label{disponer-de-un-departamento-o-servicio-especuxedficamente-para-atenciuxf3n-al-cliente-artuxedculo-4}}

\begin{itemize}
\tightlist
\item
  Entidades bajo el mismo grupo económico: \emph{podrán disponer de un
  departamento o servicio para todo el grupo.}
\end{itemize}

\hypertarget{requisitos-de-los-titulares-del-departamento-o-servicio-de-atenciuxf3n-artuxedculo-5}{%
\paragraph{\texorpdfstring{Requisitos de los titulares del departamento
o servicio de atención (\textbf{Artículo
5}):}{Requisitos de los titulares del departamento o servicio de atención (Artículo 5):}}\label{requisitos-de-los-titulares-del-departamento-o-servicio-de-atenciuxf3n-artuxedculo-5}}

\begin{itemize}
\tightlist
\item
  \textbf{Requisitos Personales:}

  \begin{itemize}
  \tightlist
  \item
    Honorabilidad comercial y profesional:

    \begin{itemize}
    \tightlist
    \item
      Aquellos que dispongan de una trayectoria personal de respeto a
      las leyes mercantiles, regulaciones de actividad económica, y
      prácticas comerciales y financieras.
    \end{itemize}
  \item
    Conocimiento y experiencia adecuada:

    \begin{itemize}
    \tightlist
    \item
      Aquellos que hayan desempeñado funciones relacionadas con la
      actividad financiera de la entidad.
    \end{itemize}
  \end{itemize}
\item
  \textbf{Designación}

  \begin{itemize}
  \tightlist
  \item
    Los titulares serán designados por el consejo de administración o
    equivalente, o la dirección general de la sucursal.
  \end{itemize}
\item
  \textbf{Comunicación de Designación}

  \begin{itemize}
  \tightlist
  \item
    Al \texttt{Comisionado(s)} para la Defensa del Cliente de Servicios
    Financieros y autoridades supervisadoras.
  \end{itemize}
\end{itemize}

\hypertarget{medidas-oportunas-para-artuxedculo-6}{%
\paragraph{\texorpdfstring{Medidas oportunas para (\textbf{Artículo
6}):}{Medidas oportunas para (Artículo 6):}}\label{medidas-oportunas-para-artuxedculo-6}}

\begin{itemize}
\tightlist
\item
  Separar el departamento o servicio de atención al cliente de los demás
  servicios comerciales

  \begin{itemize}
  \tightlist
  \item
    Propósito: evitar conflictos de interés y garantizar la toma de
    decisiones de manera autónoma.
  \end{itemize}
\item
  Responder a los principios de rapidez, seguridad, eficacia y
  coordinación.
\item
  Garantizar conocimiento adecuado del personal al servicio de la
  normativa sobre transparencia y protección de los clientes de
  servicios financieros.
\end{itemize}

\hypertarget{aprobar-un-reglamento-para-la-defensa-del-cliente-artuxedculo-8}{%
\paragraph{\texorpdfstring{Aprobar un Reglamento para la Defensa del
Cliente (\textbf{Artículo
8}):}{Aprobar un Reglamento para la Defensa del Cliente (Artículo 8):}}\label{aprobar-un-reglamento-para-la-defensa-del-cliente-artuxedculo-8}}

\begin{itemize}
\tightlist
\item
  Regular la actividad del departamento o servicio de atención (y del
  defensor del cliente, en su caso).
\item
  \textbf{Aprobación:}

  \begin{itemize}
  \tightlist
  \item
    Por el consejo de administración o equivalente
  \end{itemize}
\item
  \textbf{Ratificación:}

  \begin{itemize}
  \tightlist
  \item
    Por la junta o asamblea general o equivalente
  \end{itemize}
\item
  \textbf{Elementos (al menos los siguientes):}

  \begin{itemize}
  \tightlist
  \item
    Duración del mandato y posibilidad de renovación (en su caso)
  \item
    Causas de incompatibilidad, inelegibilidad y cese
  \item
    Relación clara y precisa de los asuntos que le pertenece a cada uno

    \begin{itemize}
    \tightlist
    \item
      Si se atribuye el conocimiento del mismo tipo de reclamación a
      ambas (departamento de atención al cliente y defensor del
      cliente):
    \end{itemize}
  \item
    Facilitar toda información solicitada en relación con el ejercicio
    de las funciones del departamento o defensor.
  \item
    Plazo para presentar reclamaciones
  \item
    Superior a 2 años desde que el cliente tuviera conocimiento de los
    hechos
  \item
    Concreción de los trámites internos
  \end{itemize}
\item
  \textbf{Organismo de control e inspección:}

  \begin{itemize}
  \tightlist
  \item
    Verificar que el reglamento contiene la regulación necesaria
  \item
    Plazo:

    \begin{itemize}
    \tightlist
    \item
      3 meses desde la presentación de la solicitud
    \end{itemize}
  \item
    Corresponde:

    \begin{itemize}
    \tightlist
    \item
      Para las \texttt{entidades\ aseguradoras} --\textgreater{} a la
      \texttt{Dirección\ General\ de\ Seguros\ y\ Fondos\ de\ Pensiones\ del\ Ministerio\ de\ Economía}

      \begin{itemize}
      \tightlist
      \item
        Excepción: si son entidades de competencia autonómica
        --\textgreater{} al \texttt{órgano\ competente\ de\ la\ CCAA} en
        donde radique el domicilio social de la entidad.
      \end{itemize}
    \item
      Si un mismo reglamento se aplica a todas las entidades de un mismo
      grupo --\textgreater{} entidad de verificación correrá a cargo de
      la entidad dominante.
    \end{itemize}
  \end{itemize}
\end{itemize}

\hypertarget{poner-a-disposiciuxf3n-del-cliente-toda-la-informaciuxf3n-siguiente-artuxedculo-9}{%
\paragraph{\texorpdfstring{Poner a disposición del cliente toda la
información siguiente (\textbf{Artículo
9}):}{Poner a disposición del cliente toda la información siguiente (Artículo 9):}}\label{poner-a-disposiciuxf3n-del-cliente-toda-la-informaciuxf3n-siguiente-artuxedculo-9}}

\begin{itemize}
\tightlist
\item
  Existencia de un departamento de atención al cliente, y defensor del
  cliente si lo hay

  \begin{itemize}
  \tightlist
  \item
    Especificar dirección postal y electrónica
  \end{itemize}
\item
  Obligación de la entidad de atender y resolver quejas en un plazo de 2
  meses
\item
  Referencia al Comisionado o Comisionados para la Defensa del Cliente
  de Servicios Financieros que correspondan

  \begin{itemize}
  \tightlist
  \item
    Con dirección postal y electrónica
  \item
    La necesidad de agotar la vía del departamento o servicio de
    atención al cliente o del defensor del cliente
  \end{itemize}
\item
  Reglamento de funcionamiento
\item
  Referencias a la normativa de transparencia y protección del cliente
  de servicios financieros.
\item
  Las decisiones expresarán la facultad que asiste al reclamante para
  acudir al Comisionado para la Defensa del Cliente de Servicios
  Financieros en caso de disconformidad.
\end{itemize}

\hypertarget{aseguradoras-podruxe1n}{%
\subsection{2.2 Aseguradoras podrán}\label{aseguradoras-podruxe1n}}

\hypertarget{designar-un-defensor-del-cliente-para-que-artuxedculo-4.2.}{%
\paragraph{\texorpdfstring{Designar un defensor del cliente para que
(\textbf{Artículo
4.2.}):}{Designar un defensor del cliente para que (Artículo 4.2.):}}\label{designar-un-defensor-del-cliente-para-que-artuxedculo-4.2.}}

\begin{itemize}
\tightlist
\item
  Atienda y resuelva las reclamaciones.
\item
  Promueva el cumplimiento de la normativa de transparencia y
  protección, y de las buenas prácticas.

  \begin{itemize}
  \tightlist
  \item
    El defensor del cliente se puede efectuar conjuntamente con otras
    entidades.
  \end{itemize}
\end{itemize}

\hypertarget{si-se-designa-un-defensor-del-cliente-deberes-artuxedculo-7}{%
\paragraph{\texorpdfstring{Si se designa un defensor del cliente
--\textgreater{} Deberes (\textbf{Artículo
7}):}{Si se designa un defensor del cliente --\textgreater{} Deberes (Artículo 7):}}\label{si-se-designa-un-defensor-del-cliente-deberes-artuxedculo-7}}

\begin{itemize}
\tightlist
\item
  Requisitos Personales (igual que los titulares del departamento):

  \begin{itemize}
  \tightlist
  \item
    Honorabilidad comercial y profesional
  \item
    Conocimiento y experiencia adecuada.
  \end{itemize}
\item
  Designación

  \begin{itemize}
  \tightlist
  \item
    Los defensores serán designados por el consejo de administración o
    equivalente, o la dirección general de la sucursal
  \item
    La designación puede ser ratificada posteriormente por la junta o
    asamblea general o equivalente.
  \end{itemize}
\item
  Comunicación de Designación

  \begin{itemize}
  \tightlist
  \item
    Al \texttt{Comisionado(s)} para la Defensa del Cliente de Servicios
    Financieros y autoridades supervisadoras.
  \end{itemize}
\item
  Actuación:

  \begin{itemize}
  \tightlist
  \item
    Con independencia con respecto de la entidad
  \item
    Total autonomía en cuanto a los criterios y directrices a aplicar
  \end{itemize}
\item
  Decisiones vinculantes

  \begin{itemize}
  \tightlist
  \item
    Las decisiones favorables al reclamante vincularán a la entidad.
  \end{itemize}
\end{itemize}

\hypertarget{procedimiento-capuxedtulo-iii}{%
\section{\texorpdfstring{3. Procedimiento (\emph{Capítulo
III})}{3. Procedimiento (Capítulo III)}}\label{procedimiento-capuxedtulo-iii}}

\hypertarget{tiempo-artuxedculo-10.3.}{%
\paragraph{\texorpdfstring{Tiempo (\textbf{Artículo
10.3.}):}{Tiempo (Artículo 10.3.):}}\label{tiempo-artuxedculo-10.3.}}

\begin{itemize}
\tightlist
\item
  Los departamentos o servicios de atención al cliente y los defensores
  del cliente, dispondrán de un plazo de dos meses desde la presentación
  de la queja para dictar un pronunciamiento.

  \begin{itemize}
  \tightlist
  \item
    El cómputo del plazo máximo de terminación comenzará a contar desde
    la presentación de la queja o reclamación en el departamento o
    servicio de atención al cliente o defensor del cliente (Artículo
    12.1)
  \item
    A partir de la finalización de dicho plazo el reclamante puede
    acudir al Comisionado para la Defensa del Cliente de Servicios
    Financieros.
  \end{itemize}
\end{itemize}

\hypertarget{forma-y-lugar-artuxedculo-11.1.-y-11.3.}{%
\paragraph{\texorpdfstring{Forma y lugar (\textbf{Artículo 11.1. y
11.3.}):}{Forma y lugar (Artículo 11.1. y 11.3.):}}\label{forma-y-lugar-artuxedculo-11.1.-y-11.3.}}

\begin{itemize}
\tightlist
\item
  Personalmente o mediante representación
\item
  En soporte papel o por medios informáticos, electrónicos o telemáticos
  (deberá ajustarse a la Ley 59/2003, de 19 de diciembre, de firma
  electrónica)
\item
  Ante los departamentos o servicios de atención al cliente, ante el
  defensor del cliente, en cualquier oficina abierta al público de la
  entidad, en la dirección de correo electrónico
\end{itemize}

\hypertarget{contenido-artuxedculo-11.2.}{%
\paragraph{\texorpdfstring{Contenido (\textbf{Artículo
11.2.}):}{Contenido (Artículo 11.2.):}}\label{contenido-artuxedculo-11.2.}}

\begin{itemize}
\tightlist
\item
  Nombre, apellidos y domicilio del interesado, DNI para las personas
  físicas y datos referidos a registro público para las jurídicas.
\item
  Motivo de la queja o reclamación
\item
  Oficina u oficinas, departamento o servicio donde se hubieran
  producido los hechos objeto
\item
  El reclamante no tiene conocimiento de que la materia objeto de la
  queja o reclamación está siendo sustanciada a través de un
  procedimiento administrativo, arbitral o judicial.
\item
  Lugar, fecha y firma.
\item
  Las pruebas documentales que obren en su poder en que se fundamente su
  queja o reclamación.
\end{itemize}

\hypertarget{admisiuxf3n-artuxedculo-12}{%
\paragraph{\texorpdfstring{Admisión (\textbf{Artículo
12}):}{Admisión (Artículo 12):}}\label{admisiuxf3n-artuxedculo-12}}

\begin{itemize}
\tightlist
\item
  La queja o reclamación ésta será remitida internamente a la entidad
  correspondiente (el departamento o servicio de atención al cliente o
  al defensor del cliente).
\item
  Deberá informarse al reclamante sobre la instancia competente para
  conocer su queja o reclamación.
\item
  Se deberá acusar recibo por escrito, dejar constancia de la fecha de
  presentación, y se procederá a la apertura de expediente.
\item
  Si hay un error en la queja o reclamación, se requerirá al firmante
  para completar la documentación remitida en el plazo de diez días
  naturales
\item
  Sólo podrá rechazarse la admisión a trámite de las quejas y
  reclamaciones en los casos siguientes:

  \begin{itemize}
  \tightlist
  \item
    Cuando se omitan datos esenciales para la tramitación
  \item
    Cuando sea competencia de los órganos administrativos, arbitrales o
    judiciales, o la misma se encuentre pendiente de resolución o
    litigio o haya sido ya resuelto en aquellas instancias
  \item
    Cuando los hechos, razones y solicitud no se refieran a operaciones
    concretas o no se ajusten a los requisitos establecidos en esta
    Orden.
  \item
    Cuando se formulen quejas o reclamaciones que reiteren otras
    anteriores resueltas, presentadas por el mismo cliente en relación a
    los mismos hechos.
  \item
    Cuando hubiera transcurrido el plazo para la presentación de quejas
    y reclamaciones.
  \end{itemize}
\item
  Cuando se entienda no admisible la queja o reclamación el tiene un
  plazo de diez días naturales para que presente sus alegaciones.
\item
  Cuando el interesado hubiera contestado y se mantengan las causas de
  inadmisión, se le comunicará la decisión final adoptada.
\end{itemize}

\hypertarget{tramitaciuxf3n-artuxedculo-13}{%
\paragraph{\texorpdfstring{Tramitación (\textbf{Artículo
13}):}{Tramitación (Artículo 13):}}\label{tramitaciuxf3n-artuxedculo-13}}

\begin{itemize}
\tightlist
\item
  En el curso de tramitación, se podrá recabar cuantos datos,
  aclaraciones, informes o elementos de prueba consideren oportunos.
\end{itemize}

\hypertarget{allanamiento-y-desistimiento-artuxedculo-14}{%
\paragraph{\texorpdfstring{Allanamiento y desistimiento
(\textbf{Artículo
14}):}{Allanamiento y desistimiento (Artículo 14):}}\label{allanamiento-y-desistimiento-artuxedculo-14}}

\begin{itemize}
\tightlist
\item
  Rectificación de situación por parte de la entidad como consecuencia
  de la queja: comunicarlo a la instancia competente y justificarlo
  documentalmente --\textgreater{} archivo de la queja
\item
  Desistimiento de la queja: en cualquier momento.
\end{itemize}

\hypertarget{finalizaciuxf3n-y-notificaciuxf3n-artuxedculo-15}{%
\paragraph{\texorpdfstring{Finalización y notificación (\textbf{Artículo
15}):}{Finalización y notificación (Artículo 15):}}\label{finalizaciuxf3n-y-notificaciuxf3n-artuxedculo-15}}

\begin{itemize}
\tightlist
\item
  Plazo para finalizar expediente: 2 meses desde que se presentase la
  queja
\item
  Decisión: motivada y con conclusiones claras
\item
  Notificación: en el plazo de 10 días naturales desde su fecha
\end{itemize}

\hypertarget{relaciuxf3n-con-los-comisionados-artuxedculo-16}{%
\paragraph{\texorpdfstring{Relación con los Comisionados
(\textbf{Artículo
16}):}{Relación con los Comisionados (Artículo 16):}}\label{relaciuxf3n-con-los-comisionados-artuxedculo-16}}

\begin{itemize}
\tightlist
\item
  Deber de atender los requerimientos que los Comisionados puedan
  efectuar a las entidades.
\item
  Acuerdos necesarios para facilitar la transmisión de datos y
  documentos entre los Comisionados y las entidades.
\end{itemize}

\hypertarget{informe-anual-capuxedtulo-iv}{%
\section{\texorpdfstring{4. Informe Anual (\emph{Capítulo
IV})}{4. Informe Anual (Capítulo IV)}}\label{informe-anual-capuxedtulo-iv}}

\hypertarget{contenido-artuxedculo-17}{%
\paragraph{\texorpdfstring{Contenido (\textbf{Artículo
17}):}{Contenido (Artículo 17):}}\label{contenido-artuxedculo-17}}

\begin{itemize}
\tightlist
\item
  En el primer trimestre de cada año:

  \begin{itemize}
  \tightlist
  \item
    Los departamentos y, en su caso, los defensores, presentarán al
    consejo de administración o equivalente un informe explicativo del
    desarrollo de su función.
  \end{itemize}
\item
  Contenido mínimo:

  \begin{itemize}
  \tightlist
  \item
    Resumen estadístico de las quejas y reclamaciones atendidas
  \item
    Resumen de las decisiones dictadas y su carácter favorable o
    desfavorable
  \item
    Criterios generales en las decisiones
  \item
    Recomendaciones o sugerencias
  \end{itemize}
\item
  La memoria anual de las entidades debe contener al menos un resumen
  del informe.
\end{itemize}

\end{document}
